\documentclass{article}

\usepackage{amsmath,mathtools}
\usepackage{bm,extarrows,ulem}
\usepackage{mathrsfs}
\usepackage{geometry,graphicx,color}
\geometry{centering,scale=0.8}

\newcommand{\sect}{\section}
\newcommand{\subsec}{\subsection}

\newcommand{\be}{\begin{equation}}
\newcommand{\ee}{\end{equation}}
\newcommand{\bea}{\begin{eqnarray}}
\newcommand{\eea}{\end{eqnarray}}
\newcommand{\ba}{\begin{array}}
\newcommand{\ea}{\end{array}}
\newcommand{\bs}{\be\begin{split}}

\newcommand{\dif}{\,\mathrm{d}}
\newcommand{\p}{\partial}
\newcommand{\1}{\left}
\newcommand{\2}{\right}
\newcommand{\ma}{\mathcal}
\newcommand{\la}{\langle}
\newcommand{\ra}{\rangle}

\newcommand{\m}{\mu}
\newcommand{\n}{\nu}
\newcommand{\al}{\alpha}
\newcommand{\bet}{\beta}
\newcommand{\lam}{\lambda}
\newcommand{\sig}{\sigma}
\newcommand{\ep}{\epsilon}
\newcommand{\om}{\omega}
\newcommand{\del}{\delta}


\title{Notes on Statistical Field Theory}
\author{Lecturer: Prof. Dr. Dirk H. Rischke\\Notes Taker: Gui-Rong Liang}

\begin{document}
\maketitle
\tableofcontents

\newpage

\sect{Thermodynamics \& Statistical Mechanics}
Closed system, Volume $V$, in thermodynamic equilibrium\\
Conserved quantities: energy $E$, particle no. $N$.\\
Comments: a) particle and anti-particle pairs can be created out of vacuum, only net particle no. or quantum numbers (e.g. net electric charge $N=N^+-N^-$) are conserved.\\
b) different particle species, different conserved quantum numbers, $N_1, N_2, ..., N_M$.\\

Thermodaynamics\\

For closed system of volume $V$ and $M$ different conserved quantum numbers,
\be
\dif E=T\dif S-p\dif V+\sum_i^M\m_i \dif N_i.
\ee

Energy is an extensive quantity, it scales with volume $V$. If $V\rightarrow \al V$, then $E(\al S, \al V, \{\al N_i\})=\al E(S, V, \{N_i\})$. Now let $\al=1+\ep$, where $|\ep|<<1$, we have
\bs
E(S, V, \{N_i\})+\ep E(S, V, \{N_i\}) &=E(S+\ep S, V+\ep V, \{N_i+\ep N_i\})\\
&=E(S, V, \{N_i\})+\ep\1[\frac{\p E}{\p S}\bigg|_{V,\{N_i\}}S+\frac{\p E}{\p V}\bigg|_{S,\{N_i\}}V+\sum_i\frac{\p E}{\p N_i}\bigg|_{S,V}N_i\2]\\
&=E(S, V, \{N_i\})+\ep\1[TS-pV+\sum_i\m_iN_i\2],
\end{split}\ee
where in the first line we just plug in $\al=1+\ep$, in the second line we used the Taylor expansion and only kept the first-order term, and in the third line we used the partial thermodynamic relations. Comparing LHS and RHS, we immediately get
\be
E(S, V, \{N_i\})=TS-pV+\sum_i\m_iN_i.
\ee
This is the \textit{fundamental relation of thermodynamics}.\\
By taking its total derivative we have
\be
\dif E=T\dif S+S\dif T-p\dif V-V\dif p+\sum_i^M(\m_i \dif N_i+N_i\dif \m_i) \implies 0= S\dif T-V\dif p+\sum_i^M N_i\dif \m_i,
\ee
where we cancel terms due to the first law. This is the \textit{Gibbs-Duham relation}.\\

Statistical Mechanics\\

micro-canonical partition function: $\Xi(E,V,\{N_i\})$ = numbers of all possible micro-states of the system.\\
relation with thermodynamics: $S(E,V,\{N_i\})=\ln\Xi(E,V,\{N_i\})$\\
microscopic density of states is
\be
\sig(E,V,\{N_i\})=\frac{\Xi(E,V,\{N_i\})}{\dif E}=\sum_{\ma R}\nolimits'\del(E-E_{\ma R}),
\ee
where $\frac{\Xi(E,V,\{N_i\})}{\dif E}$ denotes number of micro-states per energy interval, and $\ma R$ denotes all micro-states, and prime for given $V$ and $\{N_i\}$.\\
Calculation of equation of state $S(E,V,\{N_i\})$ is possible if $\Xi(E,V,\{N_i\})$ is known, but generally this is very hard to compute.\\

Canonical ensemble:\\
Legendre transform of $E$ with respect to $S$: 
\be
F=E-TS=E-\frac{\p E}{\p S}
\ee
Canonical partition function: Laplace transform of $\sig(E,V,\{N_i\})$,
\be
Z(T,V,\{N_i\})=\int_0^\infty \dif E \, e^{-E/T} \sig(E,V,\{N_i\})=\int_0^\infty \dif E \, e^{-E/T}\sum_{\ma R}\nolimits'\del(E-E_{\ma R})=\sum_{\ma R}\nolimits'e^{-E_{\ma R}/T}
\ee
relation to thermodynamics: $F(T,V,\{N_i\})=-T\ln Z(T,V,\{N_i\})$.\\

Grand canonical ensemble:\\
Legendre transform of $F$ with respect to $N_i$: 
\be
\Omega=F-\sum_i^M \m_i N_i=E-TS-\sum_i^M \m_i N_i=-pV
\ee
and
\be
\dif \Omega=-S\dif T-p\dif V-\sum_i^M N_i\dif\m_i
\ee
where 
\be 
p=-\frac{\p\Omega}{\p V}=-\frac{\Omega}{V} 
\ee
which implies $p=p(T,\{\m_i\})$ does not depend on $V$.

Grand canonical partition function is the Laplace transform of the canonical partition function,
\bs
Z(T,V,\{\m_i\})&=\sum_{N_1} e^{\frac{\m_1 N_1}{T}}....\sum_{N_M} e^{\frac{\m_M N_M}{T}} Z(T,V,\{N_i\})\\
&=\sum_{N_1}....\sum_{N_M}\sum_{\ma R}\nolimits'e^{\sum\limits_{i=1}^M\frac {\m_iN_i} T}e^{-E_{\ma R}/T}\\
&=\sum\nolimits_{\ma R}e^{-\frac 1 T(E_{\ma R}-\sum\limits_{i=1}^M\m_iN_i)},
\end{split}\ee
where the unprimed sum denotes no restriction except fixing the volume.\\
relation to thermodynamics: 
\bs
\Omega(T,V,\{\m_i\})&=-T\ln Z(T,V,\{\m_i\})\\
p(T,\{\m_i\})&=\frac T V\ln Z(T,V,\{\m_i\}).
\end{split}\ee















\sect{Quantum mechanics \& Quantum Field Theory}
completeness relation:
\be
\int \prod_{\vec x} \dif \phi_0(\vec x) |\phi_0\ra\la\phi_0|=1
\ee
orthogonality:
\be
\la\phi_a|\phi_b\ra=\prod_{\vec x} \del(\phi_a(\vec x)-\phi_b(\vec x))\equiv\del[\phi_a-\phi_b],
\ee
where the new delta notation is called the ``functional $\del$-function".
\sect{Fundamental Integral Representation of the Partition Function}
\subsec{The Partition Function as Fundamental Integral}
\subsec{Neutral Scalar Fields}
\subsec{Charged Scalar Fields: Bose-Einstein Condensation}
\subsec{Fermion Fields}
\subsec{Massive Vector Fields}
\subsec{The Electro-magnetic Fields}

\sect{The Partition Function in Perturbation Theory}
\subsec{Expansion in Powers of the Interaction}
\subsec{$\phi^4$ Theory}
\subsec{Yukawa Theory}
\subsec{QED}
\subsec{QCD}

\sect{Resummation Techniques}
\subsec{IR Divergences \& Breakdown of Naive Perturbation Series}
 \subsec{1PI Effective Action}
 \subsec{2PI Effective Action (C-J-T Formalism)}
 \subsec{Restoration of Spontanously ?? Symmetries}
 \subsec{Symmetry Restoration in H\&H-F Approximation}
 \subsec{Fermions: the N-J-L Model}
 \subsec{Superfluidity \& Superconductivity}
 \subsec{Gauge Theories: Hard Thermal Loop \& Hard Dense Loop (Random Phase Approximation)}
 \subsec{The Function Renormalization Group}


\end{document}
