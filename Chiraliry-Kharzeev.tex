\documentclass{article}

\usepackage{amsmath,mathtools,amssymb}
\usepackage{bm,extarrows,ulem,cancel}
\usepackage{mathrsfs}
\usepackage{geometry,graphicx,color}
\geometry{centering,scale=0.8}

\newcommand{\sect}{\section}
\newcommand{\subsec}{\subsection}

\newcommand{\be}{\begin{equation}}
\newcommand{\ee}{\end{equation}}
\newcommand{\bea}{\begin{eqnarray}}
\newcommand{\eea}{\end{eqnarray}}
\newcommand{\ba}{\begin{array}}
\newcommand{\ea}{\end{array}}
\newcommand{\bs}{\be\begin{split}}

\newcommand{\dif}{\,\mathrm{d}}
\newcommand{\p}{\partial}
\newcommand{\1}{\left}
\newcommand{\2}{\right}
\newcommand{\ma}{\mathcal}
\newcommand{\la}{\langle}
\newcommand{\ra}{\rangle}

\newcommand{\m}{\mu}
\newcommand{\n}{\nu}
\newcommand{\al}{\alpha}
\newcommand{\bet}{\beta}
\newcommand{\lam}{\lambda}
\newcommand{\sig}{\sigma}
\newcommand{\ep}{\epsilon}
\newcommand{\om}{\omega}
\newcommand{\del}{\delta}
\newcommand{\Del}{\Delta}

\title{Notes on Chirality by Kharzeev\\
--- Fudan Summer School}
\author{Gui-Rong Liang}

\begin{document}
\maketitle
\tableofcontents

\newpage

\section{Chirality in Classical Mechanics}
\section{Chirality in Quantum Mechanics}
The relativistic relation is given by
\be
E^2=\vec p^2+m^2,
\ee
thus the the equation of the wave function will be 
\be
\hat H^2\psi=(\hat p^2+m^2)\psi,
\ee
when we make the following substitution:
\be\1\{\begin{split}
\hat H &\rightarrow i\p_t\\
\hat p &\rightarrow -i\p_x,
\end{split}\2.\ee
we have the Klein-Gorden equation
\be
(\Box+m^2)\psi\equiv(\eta^{\m\n}\p_\m\p_\n+m^2)\psi=(\p_t^2-\p_x^2+m^2)\psi=0.
\ee
If we want a first-order equation, we must take the ``square root" of the above operator, since $\vec p$ and $m$ are the only two ingredients we can include, we assume the Hamiltonian to be
\be
H=\vec\al\cdot\vec p+\bet m,
\ee
where $\al$ is a vector to make the Hamiltonian a scalar and $\bet$ is a number.\\
By taking the square of it and comparing it to the relativistic relation, we have
\be\begin{split}
\vec p^2+m^2=H^2&=(\al^i p_i+\bet m)(\al^j p_j+\bet m)\\
&={\al^i}^2 p_i^2+(\al^i\al^j+\al^j\al^i)p_i p_j\big|_{i>j}+(\al^i\bet+\bet\al^i)p_i +\bet^2m^2,
\end{split}\ee
then we find that
\be\1\{\begin{split}
{\al^i}^2&=\bet^2=1\\
\{\al^i,\al^j\}&\equiv\al^i\al^j+\al^j\al^i=0\\
\{\al^i,\bet\}&\equiv\al^i\bet+\bet\al^i=0,
\end{split}\2.\ee
from the anti-commutation relations above, we may notice that $\al^i$ and $\bet$ cannot be numbers, but they have to be matrices. Further, we could figure the trace of these matrices as
\be\1\{\begin{split}
tr(\al^i)=tr(\bet^2\al^i)=tr(\bet\al^i\bet)=-tr(\bet^2\al^i)=-tr(\al^i)&\implies tr(\al^i)=0\\
tr(\bet)=tr({\al^i}^2\bet)=tr(\al^i\bet\al^i)=-tr({\al^i}^{2}\bet)=-tr(\bet)&\implies tr(\bet)=0.
\end{split}\2.\ee
Since the trace is the sum of eigenvalues, the traceless property suggests that these matrices have equal-magnitude but opposite-sign eigenvalues, and they have even dimensions.\\

The simplest (even) matrices are $2\times 2$, and they can be represented by superpositions of the three Pauli matrices $\sig^i$ and the unit matrix $I$. If we choose $\al^i\rightarrow\sig^i$ and $\bet\rightarrow I$, we could find the anti-commutation relation $\{\al^i,\bet\}=0$ cannot hold, which implies $2\times2$ matrices may not be possible. (This argument is not fully convincing, and to be proved more strictly in the future.) So the next lowest even-dimension matrices are $4\times4$.\\

There are several representations of the $\al^i$ and $\bet$. The Pauli-Dirac representation is given by
\be
\al^i=\1(\ba{cc}O&\sig^i\\\sig^i&O\ea\2), \quad \bet=\1(\ba{cc}I&O\\O&-I\ea\2);
\ee
while the Weyl/Chiral representation is denoted as 
\be
\al^i=\1(\ba{cc}-\sig^i&O\\O&\sig^i\ea\2), \quad \bet=\1(\ba{cc}O&I\\I&O\ea\2).
\ee
Parallel, the state $\psi$ will be a column vector with four components, and its Hermitian conjugate a row vector with four components,
\be
\psi=\1(\ba{c}\psi_1\\\psi_2\\\psi_3\\\psi_4\ea\2), \quad \psi^\dagger=\1(\ba{cccc}\psi_1^*&\psi_2^*&\psi_3^*&\psi_4^*\ea\2).\\
\ee

Now that we have found a set of reasonable coeeficient matrices, we come back to our equation of wave function, which is now
\be\label{al}
(-i\al^j\p_j+\bet m)\psi=i\p_0\psi,
\ee
the complex conjugate is given by
\be
\psi^\dagger(i\overleftarrow{\p_j}\al^j+\bet m)=-i\p_0\psi^\dagger.
\ee
Multiplying the first equation by $\psi^\dagger$ on the left and second equation by $\psi$ on the right and subtracting them, we will have the continuity equation,
\be
i\frac\p{\p t}\1(\psi^\dagger\psi\2)=-i\1[\psi^\dagger\al^i(\p_j\psi)+(\p_j\psi^\dagger)\al^i\psi\2]=-i\p_j(\psi^\dagger\al^i\psi),
\ee
so the density and flow are given by
\be\1\{\begin{split}
\rho&=\psi^\dagger\psi\\
j^i&=\psi^\dagger\al^i\psi.
\end{split}\2.\ee
We proceed to modify our equation into a more neat form. Multiplying equation \eqref{al} by $\bet$ on the left and bringing in the new symbols,
\be
\gamma^\m=(\bet,\bet \al^i), \qquad {\gamma^\m}^\dagger=(\bet, \al^i\bet)=\gamma^0 \gamma^\m \gamma^0
\ee
we get
\be
(-i\gamma^j\p_j+m)\psi=i\gamma^0\p_0\psi \quad\implies\quad (i\gamma^\m\p_\m-m)\psi=0,
\ee
where the mass $m$ should be understood as $4\times4$ matrix with diagonal elements $m$. This is the famous \textit{Dirac equation}.\\
Take the complex conjugate we have
\be
\psi^\dagger (i\gamma^0 \gamma^\m \gamma^0\p_\m-m)=0.
\ee
By further introducing a new symbol as $\bar\psi=\psi^\dagger\gamma^0$, and left multiplying the Dirac equation by $\bar\psi$ and right multiplying the complex conjugate Dirac equation by $\gamma^0 \psi$ and subtracting them, we have
\be
\p_\m j^\m\equiv\p_\m (\bar\psi\gamma^\m\psi)=0,
\ee
where 
\be\1\{\begin{split}
j^0&=\psi^\dagger\gamma^0\gamma^0\psi=\psi^\dagger\psi\\
j^i&=\psi^\dagger\gamma^0\gamma^i\psi=\psi^\dagger \al^i \psi,
\end{split}\2.\ee
which are consistent with the previous result.\\

It is useful and convenient to introduce another symbol represented as
\be
\gamma^5\equiv i\gamma^0\gamma^1\gamma^2\gamma^3.
\ee
By using the well-known anti-commutation relations between $\gamma$ matrices
\be
\{\gamma^\m,\gamma^\n\}=2\eta^{\m\n},
\ee
we can easily prove the following properties of $\gamma^5$,
\be\1\{\begin{split}
&\{\gamma^5,\gamma^\m\}=0\\
&(\gamma^5)^\dagger=\gamma^5, \quad (\gamma)^2=1.
\end{split}\2.\ee
We further introduce the left and right hand projection operator as
\be
P_{L,R}\equiv\frac1 2 (1\mp\gamma^5),
\ee
and can easily prove that
\be\1\{\begin{split}
&P_{L,R}\gamma^\m=\gamma^\m P_{R,L}\\
&P_L P_R=P_R P_L=0\\
&P_{L,R}^2=P_{L,R};
\end{split}\2.\ee
then we denote states acted by the projection operators as
\be
\psi_{L,R}\equiv P_{L,R} \psi,
\ee
and again be able to prove that
\be
\bar\psi_{L,R}\equiv\psi_{L,R}^\dagger\gamma^0=\bar\psi P_{R,L}.
\ee
Now we ready to separate our conserved current into left and right hand parts:
\be\begin{split}
j^\m&=\bar\psi\gamma^\m\psi=(\bar\psi_L+\bar\psi_R)\gamma^\m(\psi_L+\psi_L)\\
&=\bar\psi_L\gamma^\m\psi_L+\bcancel{\bar\psi_L\gamma^\m\psi_R}+\bcancel{\bar\psi_R\gamma^\m\psi_L}+\bar\psi_R\gamma^\m\psi_R\\
&\equiv j^\m_L+j^\m_R,
\end{split}\ee
where the cross terms vanish due to the following proof,
\be
\bar\psi_L\gamma^\m\psi_R=\bar\psi P_R\gamma^\m P_R \psi=\bar\psi \gamma^\m  P_L P_R \psi=0,\quad \text{etc},
\ee
and it's more intriguing to write the non-vanishing terms as
\be
j^\m_{L,R}\equiv \bar\psi_{L,R}\gamma^\m\psi_{L,R}=\bar\psi P_{R,L}\gamma^\m\psi_{L,R}=\bar\psi \gamma^\m P_{L,R} \psi_{L,R}=\bar\psi \gamma^\m \psi_{L,R}=\frac 1 2 \bar\psi \gamma^\m (1\mp\gamma^5)\psi,
\ee
thus it would be easy to recover the original current and even construct a new current as
\be\1\{\begin{split}
j^\m&= j^\m_L+j^\m_R=\bar\psi\gamma^\m\psi\\
j^\m_A&= j^\m_R-j^\m_L=\bar\psi\gamma^\m\gamma^5\psi,
\end{split}\2.\ee
where the new current $j^\m_A$ is called the axial current or chiral current.\\

Consider the Lagrangian of the Dirac equation,
\be\begin{split}
\ma L&=\bar\psi(i\gamma^\m \p_\m-m)\psi\\
&=\bar\psi_L(i\gamma^\m \p_\m-m)\psi_L+\bar\psi_R(i\gamma^\m \p_\m-m)\psi_R+m(\bar\psi_L \psi_R+\bar\psi_R \psi_L),
\end{split}\ee
where the cross term between left and right hand only comes from the mass term. So when it is massless, there will be no mixing of the left and right.\\





Then we work in Pauli-Dirac representation and first explore the solution of the Dirac equation in the particle's rest frame, where the momentum is set to be zero, thus the equation becomes
\be
i\frac\p{\p t}\psi=\1(\ba{cc}m&O\\O&-m\ea\2)  \psi,
\ee
so the first two components are proportional to $e^{-imt}$ and energy $E=m$ which is positive, and the second two components are proportional to $e^{imt}$ and energy $E=-m$ which is negative.

Then we explore the solution in the moving frame where the momentum is not zero, now we have the eigenvector equation,
\be
H\1(\ba{c}\psi_A\\\psi_B\ea\2)=\1(\ba{cc}m&\vec\sig\cdot\vec p\\\vec\sig\cdot\vec p&-m\ea\2) \1(\ba{c}\psi_A\\\psi_B\ea\2)=E\1(\ba{c}\psi_A\\\psi_B\ea\2),
\ee
the two equations are
\be\1\{\begin{split}
(\vec\sig\cdot\vec p)\psi_B=(E-m)\psi_A\\
(\vec\sig\cdot\vec p)\psi_A=(E+m)\psi_B,
\end{split}\2.\ee
which suggests that the positive part and the negative part are not independent.\\
Now that $\psi_A$ is a two-component vector, assume $\psi_A$ has two solutions
\be\1\{\begin{split}
\psi_A^{(1)}\sim\chi^{(1)}=\1(\ba{c}1\\0\ea\2)\\
\psi_A^{(2)}\sim\chi^{(2)}=\1(\ba{c}0\\1\ea\2),
\end{split}\2.\ee
then we have
\be
\psi_B^{(s)}=\frac{\vec\sig\cdot\vec p}{E+m}\psi_A^{(s)}, \qquad\psi_A^{(s)}\sim\chi^{(s)}.
\ee
When $p<<m$, $\psi_B\sim0$, it decouples with $\psi_A$;\\
when $p>>m$, $\frac{\vec p}{E+m}\rightarrow \hat e_p$, where $\hat e_p$ is the unit vector along the direction of $\vec p$, so $\psi_B=(\vec\sig\cdot\hat e_p) \chi^{(s)}$, it does not decouple with $\psi_A$, and $(\vec\sig\cdot\hat e_p)$ is the projection of the spin along the momentum.

Then we consider the massless case where $m=0$, now the Hamiltonian is given by 
\be
H=\1(\ba{cc}O&\vec\sig\cdot\vec p\\\vec\sig\cdot\vec p&O\ea\2),
\ee
and then we can easily construct a matrix which commutes with the Hamiltonian as
\be
\vec\Sigma\cdot\vec p =\1(\ba{cc}\vec\sig\cdot\vec p&O\\O&\vec\sig\cdot\vec p\ea\2).
\ee
From $[\vec\Sigma\cdot\vec p,H]=0$, we see that $\vec\Sigma\cdot\vec p$ is a conserved quantity in the massless case, which we call chirality.









\section{Chirality in Gauge Theories}
\section{Chiral Magnetic Effect (CME) and Anomaly-induced transport}
\section{CME in Heavy-ion Collisions}
\section{Chiral Matters}




\end{document}