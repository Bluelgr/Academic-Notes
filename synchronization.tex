\documentclass{article}

\usepackage{amsmath}
\usepackage{bm}

\newcommand{\be}{\begin{equation}}
\newcommand{\ee}{\end{equation}}

\newcommand{\dif}{\,\mathrm{d}}

\title{Landau's clock synchronization}

\begin{document}
\maketitle
Let us write the interval, separating the space and time coordinates:
\be
\dif s^2=g_{00}\dif t^2+2g_{0i}\dif t\dif x^i+g_{ij}\d x^i\d x^j
\ee
For null signals, $ds^2=0$:
\be
0=g_{00}\dif t^2+2g_{0i}\dif t\dif x^i+g_{ij}\dif x^i\dif x^j
\ee
Solving the quadratic equation for $\dif t$:
\be
\dif t=\frac{-g_{0i}\dif x^i\pm\sqrt{(g_{0i}g_{0j}-g_{00}g_{ij})\dif x^i\dif x^j}}{g_{00}}
\ee
The two roots are:
\be
\label{t} 
\begin{split}
\dif t_1=\frac{-g_{0i}\dif x^i+\sqrt{(g_{0i}g_{0j}-g_{00}g_{ij})\dif x^i\dif x^j}}{g_{00}},\\
\dif t_2=\frac{-g_{0i}\dif x^i-\sqrt{(g_{0i}g_{0j}-g_{00}g_{ij})\dif x^i\dif x^j}}{g_{00}}
\end{split}
\ee
We should regard as simultaneous with the moment $t$ at the point A that thereading of the clock at B which is halfway between the moments of departure and return of the signal to that point, i.e. the moment
\be 
t+\dif t=t+\frac{1}{2}(\dif t_1+\dif t_2).
\ee
Substituting \eqref{t}, we thus find that the difference in the values of the ``time'' $t$ for two simultaneous events occurring at infinitely near points is given by
\be 
\dif t=\frac{-g_{0i}\dif x^i}{g_{00}}
\ee
And this is equivalent of saying that ``the `covariant differential' $\dif x_0$ between two infinitely near simultaneous events must be zero":
\be 
g_{00}\dif x^0+g_{0i}\dif x^i\equiv\dif x_0=0
\ee
This relation enables us to synchronize clocks in any infinitesimal region of space.\\

However, sychronization of clocks along a closed contour turns out to be impossible unless the following relation can be satisfied:
\be 
\oint \dif t=\oint \left(-\frac{g_{0i}}{g_{00}}\right) \dif x^i=0
\ee
or its sufficient condition is found:
\be 
g_{0i}=0
\ee
This is Landau's theory of clock synchronization.

\end{document}