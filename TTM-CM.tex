\documentclass{article}

\usepackage{amsmath,mathtools}
\usepackage{bm,extarrows,ulem}
\usepackage{mathrsfs}
\usepackage{geometry,graphicx,color}
\geometry{centering,scale=0.8}

\newcommand{\be}{\begin{equation}}
\newcommand{\ee}{\end{equation}}
\newcommand{\bea}{\begin{eqnarray}}
\newcommand{\eea}{\end{eqnarray}}
\newcommand{\ba}{\begin{array}}
\newcommand{\ea}{\end{array}}
\newcommand{\bs}{\be\begin{split}}

\newcommand{\dif}{\,\mathrm{d}}
\newcommand{\p}{\partial}
\newcommand{\1}{\left}
\newcommand{\2}{\right}
\newcommand{\ma}{\mathcal}
\newcommand{\br}{\langle}
\newcommand{\ke}{\rangle}

\newcommand{\m}{\mu}
\newcommand{\n}{\nu}
\newcommand{\al}{\alpha}
\newcommand{\bet}{\beta}
\newcommand{\lam}{\lambda}
\newcommand{\sig}{\sigma}
\newcommand{\ep}{\epsilon}
\newcommand{\del}{\delta}


\title{Notes on The Theoretical Minimum\\
--- Classical Mechanics}
\author{Gui-Rong Liang}

\begin{document}
\maketitle
\tableofcontents

\newpage

The Euler-Lagrangian equation is given by
\be
\frac \dif {\dif t} \frac{\p L}{\p \dot q_i} = \frac{\p L}{\p q_i}.
\ee

If the Lagragian $L$ remains invariant under the infinitesimal change of  coordinates,
\be
q_i\rightarrow q_i'=q_i+\ep f_i(q) \quad\text{or}\quad \delta q_i=\ep f_i(q),
\ee
we have
\bs
0=\delta L(q_i,\dot q_i)&= \sum_i \1[\frac{\p L}{\p q_i}\delta q_i+\frac{\p L}{\p \dot q_i}\delta q_i\2]\\
&= \sum_i \1[\frac{\dif}{\dif t}\1(\frac{\p L}{\p \dot q_i}\2)\delta q+ \frac{\p L}{\p\dot q_i} \frac\dif{\dif t} \bigg(\delta q_i \bigg)\2]\\
&=\sum_i \frac\dif{\dif t} \1[\frac{\p L}{\p \dot q_i}\delta q_i\2]\\
&=\ep \frac\dif{\dif t} \bigg[\sum_i  p_i f_i(q)\bigg] \\
&\equiv \ep \frac{\dif Q}{\dif t},
\end{split}\ee
where in the second row we used the Euler-Lagrangian equation, and thus the quantity $Q\equiv\sum_i  p_i f_i(q)$ is conserved. This is the Noether's Theorem, and we restate the mathematical form as follows:\\

\textit{If the Lagrangian is invariant $(\del L=0)$ under the transformations $\del q_i =\ep f_i(q)$, then the charge $Q=\sum_i  p_i f_i(q)$ is a conserved quantity.}\\

The Hamiltonian is defined as
\be
H:=\sum_i p_i \dot q_i -L=\sum_i \frac{\p L}{\p \dot q_i} \dot q_i -L,
\ee
and the Hamiltonian equations are given by
\be\1\{\begin{split}
\dot p_i&=-\frac{\p H}{\p q_i}\\
\dot q_i&=\frac{\p H}{\p p_i}.
\end{split}\2.\ee

The time derivative of a quantity $F(q_i, p_i)$ 

























\end{document}