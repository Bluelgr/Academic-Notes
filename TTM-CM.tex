\documentclass{article}

\usepackage{amsmath,mathtools}
\usepackage{bm,extarrows,ulem}
\usepackage{mathrsfs}
\usepackage{geometry,graphicx,color}
\geometry{centering,scale=0.8}

\newcommand{\be}{\begin{equation}}
\newcommand{\ee}{\end{equation}}
\newcommand{\bea}{\begin{eqnarray}}
\newcommand{\eea}{\end{eqnarray}}
\newcommand{\ba}{\begin{array}}
\newcommand{\ea}{\end{array}}
\newcommand{\bs}{\be\begin{split}}

\newcommand{\dif}{\,\mathrm{d}}
\newcommand{\p}{\partial}
\newcommand{\1}{\left}
\newcommand{\2}{\right}
\newcommand{\ma}{\mathcal}
\newcommand{\br}{\langle}
\newcommand{\ke}{\rangle}

\newcommand{\m}{\mu}
\newcommand{\n}{\nu}
\newcommand{\al}{\alpha}
\newcommand{\bet}{\beta}
\newcommand{\lam}{\lambda}
\newcommand{\sig}{\sigma}
\newcommand{\ep}{\epsilon}
\newcommand{\del}{\delta}


\title{Notes on The Theoretical Minimum\\
--- Classical Mechanics}
\author{Gui-Rong Liang}

\begin{document}
\maketitle
\tableofcontents

\newpage

\subsection{The Euler-Lagrangian equation}
The Euler-Lagrangian equation is given by
\be
\frac \dif {\dif t} \frac{\p L}{\p \dot q_i} = \frac{\p L}{\p q_i},
\ee
and the canonical momentum is defined as
\be
p_i:=\frac{\p L}{\p \dot q_i}.
\ee

\subsection{The Noether Theorom}
If the Lagragian $L$ remains invariant under the infinitesimal change of  coordinates,
\be
q_i\rightarrow q_i'=q_i+\ep f_i(q) \quad\text{or}\quad \delta q_i=\ep f_i(q),
\ee
we have
\bs
0=\delta L(q_i,\dot q_i)&= \sum_i \1[\frac{\p L}{\p q_i}\delta q_i+\frac{\p L}{\p \dot q_i}\delta q_i\2]\\
&= \sum_i \1[\frac{\dif}{\dif t}\1(\frac{\p L}{\p \dot q_i}\2)\delta q+ \frac{\p L}{\p\dot q_i} \frac\dif{\dif t} \bigg(\delta q_i \bigg)\2]\\
&=\sum_i \frac\dif{\dif t} \1[\frac{\p L}{\p \dot q_i}\delta q_i\2]\\
&=\ep \frac\dif{\dif t} \bigg[\sum_i  p_i f_i(q)\bigg] \\
&\equiv \ep \frac{\dif Q}{\dif t},
\end{split}\ee
where in the second row we used the Euler-Lagrangian equation, and thus the quantity $Q\equiv\sum_i  p_i f_i(q)$ is conserved. This is the Noether's Theorem, and we restate the mathematical form as follows:\\

\textit{If the Lagrangian is invariant $(\del L=0)$ under the transformations $\del q_i =\ep f_i(q)$, then the charge $Q=\sum_i  p_i f_i(q)$ is a conserved quantity.}\\

\subsection{The Hamiltonian mechanics}
The time derivative of the most general Lagrangian $L(q_i,\dot q_i,t)$is
\bs
\frac\dif{\dif t} L(q,\dot q,t)&=\sum_i \1[\frac{\p L}{\p q_i}\dot q_i+\frac{\p L}{\p \dot q_i}\ddot q_i\2]+\frac{\p L}{\p t}\\
&=\sum_i \1[\frac{\p L}{\p q_i}\dot q_i+\frac\dif{\dif t} \1(\frac{\p L}{\p \dot q_i}\dot q_i\2)-\frac\dif{\dif t} \1(\frac{\p L}{\p \dot q_i}\2)\dot q_i\2]+\frac{\p L}{\p t}\\
&=\frac\dif{\dif t}\sum_i  \1(\frac{\p L}{\p \dot q_i}\dot q_i\2)+\frac{\p L}{\p t}\\
&=\frac\dif{\dif t}\sum_i  p_i\dot q_i+\frac{\p L}{\p t},
\end{split}\ee
where in the second line we used the Leibniz rule for derivatives, and in the third line we used the Euler-Lagrange equation, and in the fourth line the definition of the canonical momentum. Thus we have
\be
-\frac{\p L}{\p t}=\frac \dif{\dif t}\1(\sum_i  p_i\dot q_i-L\2)\equiv \frac{\dif H}{\dif t},
\ee
which says that if the Lagrangian does not explicitly contains time, the new quantity $H$, called the Hamiltonian, is conserved.\\

We rewrite the definition of the Hamiltonian as
\be
H:=\sum_i p_i \dot q_i -L=\sum_i \frac{\p L}{\p \dot q_i} \dot q_i - L.
\ee
The change in the Hamiltonian is
\bs
\del H&=\sum_i \1(\del p_i \dot q_i+p_i \del\dot q_i -\frac{\p L}{\p q_i}\del q_i-\frac{\p L}{\p \dot q_i}\del \dot q_i\2)\\
&=\sum_i \1(\dot q_i\del p_i -\frac{\p L}{\p q_i}\del q_i\2)\\
&=\sum_i \1(\dot q_i\del p_i -\dot p_i \del q_i\2),
\end{split}\ee
where the second and the fourth term cancel due to the definition of the canonical momentum, and the third row is due to the Euler-Lagrangian equation.\\
Generally, the change in the Hamiltonian, treated as a function of multiple variables, is given by
\be
\del H(p_i,q_i)=\sum_i\1(\frac{\p H}{\p p_i} \del p_i+\frac{\p H}{\p q_i} \del q_i\2),
\ee
by matching the terms in the above two equations, we arrive at
\be\1\{\begin{split}
\dot q_i&=\frac{\p H}{\p p_i}\\
\dot p_i&=-\frac{\p H}{\p q_i},
\end{split}\2.\ee
this is the Hamilton equaitons.\\

\subsection{The Poisson Brackets}
The time derivative of a quantity $F(q_i, p_i)$ is given by
\bs
\dot F(q_i, p_i)&=\sum_i \1(\frac{\p F}{\p q_i} \dot q_i+\frac{\p F}{\p p_i} \dot p_i\2)\\
&=\sum_i \1(\frac{\p F}{\p q_i} \frac{\p H}{\p p_i}-\frac{\p F}{\p p_i} \frac{\p H}{\p q_i}\2)\\
&\equiv \{F,H\},
\end{split}\ee
where in the second row we used the Hamiltonian equations, and the notation in the third row is called the Poisson bracket.\\
The amazing thing about the above equation is that it summarizes so much. The time derivative of anything is given by the Poisson bracket of that thing with the Hamiltonian.\\
Firstly it even contains Hamilton's equations themselves,
\be\1\{\begin{split}
\dot q_i&=\{q_i,H\}\\
\dot p_i&=\{p_i,H\},
\end{split}\2.\ee
Secondly it contains the energy conservation law, only by letting $F$ to be $H$,
\be
\dot H=\{H,H\}=0.
\ee























\end{document}