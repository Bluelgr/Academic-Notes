\documentclass{article}

\usepackage{amsmath,bm,pifont}

\newcommand{\be}{\begin{equation}}
\newcommand{\ee}{\end{equation}}
\newcommand{\bea}{\begin{eqnarray}}
\newcommand{\eea}{\end{eqnarray}}
\newcommand{\ba}{\begin{array}}
\newcommand{\ea}{\end{array}}

\begin{document}
\title{Schwarzschild tunneling with HJ equation in PG-like coordinates}
\author{Gui-Rong Liang}
\maketitle
\begin{abstract}
Firstly I derive HJ equation from Lagrangian, then put it in the PG-like coordinates to calculate the emission rate of Hawking radiation. We will find some convenience of using this coordinate system during the process.
\end{abstract}
\section{From Lagrangian to HJ equation}
The Lagrangian of a free particle in a curved spacetime is given as
\be
L=\frac m 2 g_{\mu\nu}\dot{x}^\mu\dot{x}^\nu ~.
\ee
It's easy to compute the covariant four-momentum
\be
P_\mu=\frac{\partial L}{\partial \dot x^\mu}=m g_{\mu\nu}\dot x^\nu,
\ee
\renewcommand{\thefootnote}{\fnsymbol{footnote}}
So we can get the Hamiltonian\footnote{We should reexpress the velocity in terms of the momentum as $\dot x^\mu=\frac 1 m g^{\mu\nu}P_\nu ~.$} of the particle
\be
H=P_\mu \dot x^\mu-L=\frac m 2 g_{\mu\nu}\dot x^\mu \dot x^\nu=\frac1{2m} g^{\mu\nu}P_\mu P_\nu ~.
\ee
The classical HJ equation can be generalized into the curved spacetime as
\be
\frac{\partial S}{\partial \tau}+H(\dot x^\mu,\frac{\partial S}{\partial x^\mu})=0
\ee
where $\tau$ denotes the proper time of the freely moving particle,and $\frac{\partial S}{\partial x^\mu}$ is a simple substitution of $P_\mu$.\\
Thus,
\be
\frac{\partial S}{\partial\tau}+\frac1{2m}g^{\mu\nu} (\frac{\partial S}{\partial x^\mu}) (\frac{\partial S} {\partial x^\nu})=0
\ee
Note that due to the normalization condition, the Hamilton $H=-m/2$ is a conservative quantity itself, the equation above can be separated into
\be
\left\{\begin{split}
&\frac 1 {2m} g^{\mu\nu} (\frac{\partial S}{\partial x^\mu}) (\frac{\partial S} {\partial x^\nu})+\frac m 2=0 \\
&\frac{\partial S}{\partial\tau}=\frac m 2
\end{split}\right.
\ee
Generally, the proper time $\tau$ of a particle does not apparently show up in a coordinate system, so we neglect the second separated equation; moreover, in tunneling process of Hawking radiation, we care only about the imaginary part of the action $S$, while the second equation just contributes a real factor.\\
The first equation can be rearranged like
\be
g^{\mu\nu} (\frac{\partial S}{\partial x^\mu}) (\frac{\partial S} {\partial x^\nu})+m^2=0
\label{hj}
\ee
or a simpler form
\be
\fbox{$\bm{\partial^\mu S~\partial_\mu S+m^2=0$}}
\ee
This is the standard Hamilton-Jaccobi equation of a free particle with mass $m$.
\section{Schwarzschild tunneling analysis in PG-like coordinates}
The line element of a Schwarzschild metric $g_{\mu\nu}$ is expressed as
\be
\mathrm ds^2=-f\mathrm dt^2+f^{-1}\mathrm dr^2+r^2\mathrm d\Omega^2,
\ee
where $f=1-2M/r$, and $\mathrm d\Omega^2=\mathrm d\theta^2+\sin^2\theta \mathrm d\phi^2$.\\
When transformed into the PG-like coordinate family, it becomes
\be
\mathrm ds^2=-f\mathrm dT^2+2\sqrt{1-pf}\mathrm dT\mathrm dr+p\mathrm dr^2+r^2\mathrm d\Omega^2\\
\ee
or
\be
\mathrm ds^2=-\frac 1 p \mathrm dT^2+p(\mathrm dr^2+\frac 1 p \sqrt{1-pf})^2+r^2\mathrm d\Omega^2,
\ee
the transformation relationship is
\be
\mathrm dT=\mathrm dt+\frac{\sqrt{1-pf}}{f}\mathrm dr
\label{trans}
\ee
where the single parameter $p$ symbolizing the coordinate choosing, is defined as $p=1/\tilde E^2$ , with $\tilde E$ representing energy per unit mass of the particle.\footnote{More about this coordinate system family, see Ref\cite{AJM} in detail.} \\
We denote the new form of the metric as $g'_{\mu\nu}$, and get its covariant matrix form,
\be
(g'_{\mu\nu})=\left[\ba{cccc}-f&\sqrt{1-pf}&0&0\\\sqrt{1-pf}&p&0&0\\0&0&r^2&0\\0&0&0&r^2\sin^2\theta\ea\right]
\ee
and contravariant form,
\be
(g'^{\mu\nu})=\left[\ba{cccc}-p&\sqrt{1-pf}&0&0\\\sqrt{1-pf}&f&0&0\\0&0&\frac 1 {r^2}&0\\0&0&0&\frac 1 {r^2\sin^2\theta}\ea\right]
\ee
Taking the contravariant components into the equation (\ref{hj}), and considering the case of radial motions, we get the HJ equation in PG-like coordinate family\footnote{For radial motions, we neglect angular terms for the reason that \[\left\{\begin{split}
&\frac{\partial S}{\partial \theta}=P_\theta=\frac{\partial L}{\partial \dot\theta}=m g'_{\theta\theta}\dot\theta=0\\
&\frac{\partial S}{\partial \phi}=P_\phi=\frac{\partial L}{\partial \dot\phi}=m g'_{\phi\phi}\dot\phi=0.
\end{split}\right.
\]},
\be
-p(\frac{\partial S}{\partial T})^2+2\sqrt{1-pf}(\frac{\partial S}{\partial T})(\frac{\partial S}{\partial r})+f(\frac{\partial S}{\partial r})^2=-m^2.
\label{hjpg}
\ee
It seems hard to separate the variants, but we will see one of them is a conservative quantity,
\be
\begin{split}
\frac{\partial S}{\partial T}=P_T=\frac{\partial L}{\partial \dot T}&=m g'_{\scriptscriptstyle T \scriptscriptstyle T}\dot T+m g'_{\scriptscriptstyle T \scriptstyle r}\dot r=m(-f\dot T+\sqrt{1-pf}~\dot r) \\ &\stackrel{\displaystyle\text{\ding{172}}}{=}-mf~\dot t\stackrel{\displaystyle\text{\ding{173}}}{=}-m\tilde E\stackrel{\displaystyle\text{\ding{174}}}{=}-\omega,
\end{split}
\ee
the equal sign \ding{172} is the result of substituting transformation relationship (\ref{trans}) into the equation, the equal sign \ding{173} is the expression of $\tilde E$ in the original coordinate system,\footnote{In the original coordinate system, $\tilde E=g_{\mu\nu}(\frac{\mathrm dx^\mu}{\mathrm dt})(\frac{\mathrm dx^\nu}{\mathrm d\tau})=g_{tt}\frac{\mathrm dt}{\mathrm d\tau}=f~\dot t$} and $\omega$ represents the total energy of the particle.\\
So, equation (\ref{hjpg}) becomes
\be
-p~\omega^2-2\omega\sqrt{1-pf}(\frac{\partial S}{\partial r})+f(\frac{\partial S}{\partial r})^2=-m^2
\ee
Note that the definition of the parameter $p=1/\tilde E^2$ and the equal sign \ding{174} together give us a new relationship as $p~\omega^2=m^2$, which makes the equation above even simpler,
\be
(\frac{\partial S}{\partial r})(f\frac{\partial S}{\partial r}-2\omega\sqrt{1-pf})=0.
\ee
Neglecting the trivial solution, we get the partial differentiation of $S$ with respect to $r$,
\be
\partial_r S\equiv\frac{\partial S}{\partial r}=2~\omega\frac{\sqrt{1-pf}}{f}.
\ee
And the imaginary part of the action across the event horizon can be directly computed,\footnote{This can be achieved by calculating the residue, $Res(\frac{\sqrt{1-pf}}{f})|_{r=2M}=2\pi M$}
\be
ImS=-i\int_{r_{in}}^{r_{out}}2~\omega\frac{\sqrt{1-pf}}{f}~\mathrm dr=4\pi M
\ee
where $r_{in}=2M-\varepsilon$ and $r_{out}=2M+\varepsilon$, with $\varepsilon$ an infinitesimal.\\
Considering the self-gravitation effect, the result above ought to make such replacements:
\be
\left\{
\begin{split}
\omega\rightarrow\int_0^\omega\mathrm d\omega'\\
M\rightarrow M-\omega'
\end{split}
\right.
\ee
Thus, modification should be made as,
\be
\begin{split}
ImS&=4\pi\int_0^\omega (M-\omega') \mathrm d\omega'\\
&=4\pi\omega(M-\frac{\omega}{2})
\end{split}
\ee
According to WKB approximation, the tunneling rate should be like
\be
\Gamma\propto e^{-2ImS}
\ee
So finally we have our result,
\be
\Gamma\propto e^{-8\pi\omega(M-\frac{\omega}{2})}
\ee
which well accorded with previous works, being consistent with the underlying unitary theory.

\begin{thebibliography}{}
\bibitem{AJM}Regular coordinate systems for Schwarzschild and other spherical spacetimes, Karl Martel and Eric Poisson
\end{thebibliography}

\end{document}

